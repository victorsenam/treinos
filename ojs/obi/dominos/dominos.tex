O pequeno Arthur estava andando pelas ruas de Campinas e encontrou no chão uma matriz com~$n$ linhas e~$m$ colunas. Já que Arthur tinha exatamente~$nm/2$ dominós no seu bolso direito, fez a única pergunta possível para alguém em sua situação: ``De quantas maneiras eu consigo preencher esta matriz com meus dominós?''. Já que Arthur não é muito bom com computadores, pediu sua ajuda para responder esta importantíssima pergunta.

Considere que os dominós de Arthur são retângulos 2 por 1 idênticos entre si. Ele está interessado em preencher a matriz completamente com os dominós, de forma que nenhum espaço fique vazio. Arthur permite que os dominós sejam rotacionados, isto é, um dominó pode ocupar duas células adjacentes da mesma linha ou da mesma coluna da matriz.

\subsection{Entrada}


\subsection{Restrições}
\begin{itemize}
\item $1 \leq n \leq 6$
\item $1 \leq m \leq 10^{18}$
\end{itemize}

\subsection{Pontuação}
